\chapter{Introduction IAS Latex Template\label{kap:einfuehrung}}
\textbf{This introduction is in English, independent of the chosen language in the main document.}

\section{Revision History}

\begin{table}[!ht]
  \centering      
  \begin{tabularx}{\textwidth}{rlX}
  \toprule
  Revision            & Author  & Comment   \\
  \midrule
  1                   & Ralf Wunderlich   & Initial Revision \\
                      & Andreas Neyer     & \\
                      & Niklas Zimmermann & \\
  2                   & Bastian Mohr      & Added switches for language, draft, confidential \\
                      & Sebastian Strache & added some packages (e.g. SIUnitX) \\
  3                   & Martin Schleyer   & Added listing environment \\
  4                   & Bastian Mohr      & Changed RWTH logo to CD, removed title badbox \\            
  4                   & Josua Arndt       & Changed color of links in online version \\             5                   & Ralf Wunderlich   & Changed Logo, title page, exploitation rights, submission date \\ 
  5                & Josua Arndt        & Logo Switch NoLogo, removed Warnings, removed DraftMode and added hint in setting.tex for activating draft by setting Document options, changed this Tabular to tabularx, Checked document for functionality, added package eps2pdf for including eps and automatically converting to pdf, changed Hyperlink to automatically switch to Printmode and made the colors of the rectangles match the linkcolor\\
  6                & Josua Arndt        & Force text on tittlepage to be on the same place, no matter how long the Titles are ( 2 times 5 lines). Increase titelpage size. Changed Doublepagestyle to empty.\\ 
  \bottomrule
  \end{tabularx}        
   \caption{Revision History\label{tab:revision-history}}
\end{table}

For questions on how to use the template please contact Josua Arndt. A list od Changes made in the Different Revision can be found in Table~\ref{tab:revision-history}. 

\section{Important Notes}

This template should be used for all thesises at the institute. The usage of the RWTH and IAS Logo at the title page is by choice, it can be left out as a whole (but not replaced by any other logo). 
All other changes in the template need to be authorized by the supervisor.  

Read this document carefully. You may take a look in the sourcecode of beispiel.tex as it contains some useful information. This document is no introduction to \LaTeX itself, read for example \enquote{The not so short introduction to latex} \cite{Oetiker11} for an introduction. This template is tested to work with TexPoint and MikTex.

\section{Encoding}

Check these letters: � � � � � �

The encoding of these files is set to latin1, also called ISO 8859 or ANSI. Some might think Unicode (UTF8) is better but this also yields to some problems

\begin{itemize}
	\item Windows and Linux have different problems when opening the files
  \item bibtex does not support UTF8 so the bib-file is encoded in ANSI anyway (which is ugly)
  \item Yes, biber supports UTF8 but is not part of MikTex 2.8 which is currently used
\end{itemize}

If there is any issue with weird characters, change the encoding to ANSI and your done.

\section{About the template}

Check the file settings.tex. You will finde the lines given in listing \ref{lis:latex-config}. If you set \emph{PrintVersion} to \emph{true} the logos and syntax highlighting are optimized for black and white printer output. Set it to \emph{false} to generate the online version with colors. Do not print the online version, it looks really crappy.

If you have confidential information in your document you can generate a confidential or a public version by setting the switch \emph{ConfidentialVersion} to \emph{true} or \emph{false} respectively. Setting the switch \emph{IsEnglish} to \emph{true} set all language specific properties to English, \emph{false} is used for German documents.

The further parameters define author name, title of the thesis and thesis type. This information is included in the pdf meta information or the title page. Finally you can adjust the name of your BibTex file and the path where your images are located so you don't have to add it everywhere.



\begin{lstlisting}[language={TeX}, caption=Document Configuration.,label=lis:latex-config]
%\setboolean{DraftMode}{false} 
\setboolean{PrintVersion}{false} 
\setboolean{ConfidentialVersion}{false}
\setboolean{IsEnglish}{true}

\newcommand{\IASAuthor}{Nachname, Vorname}
\newcommand{\IASTitleGerman}{IAS Vorlage}
\newcommand{\IASTitleEnglish}{IAS Template}
\newcommand{\IASSubject}{Masterarbeit}% Bachelorarbeit, Studienarbeit usw.
\newcommand{\IASMatNr}{123456}
\newcommand{\IASSupervisor}{BetreuerNachname, BetreuerVorname M.Sc.}
\newcommand{\IASNumber}{D123}
\newcommand{\IASSubmissionDate}{dd.mm.yyyy}

\graphicspath{{bilder/}} %relativer Pfad, in dem die Bilder liegen

\bibliography{mybib}

\end{lstlisting}

Use the file content.tex to add your TeX files. It is recommended to use one file per chapter. If you want to have a list of abbreviations you have to execute the following command (using your documents name) after new abbreviations were added to the document:

\begin{verbatim}
  makeindex Vorlage_DA.nlo  -s nomencl.ist -o Vorlage_DA.nls
\end{verbatim}

You can add your own chapters right after the \emph{input} command for this file. 

\textbf{Do not change anything else in the main document. Usually it is not necessary to add further packages or change any setting.}




\section{Content Examples}

\subsection{Text}
New paragraphs are generated with one or more empty lines in the source code. There are two kinds of paragraphs. New line is done by using \\ which looks quite ugly.

More sections are

\subsubsection{Subsubsection}
You should not use anything below this. This will not show up in the TOC.

\paragraph{Paragraph}
Usually this and the following is useless

\subparagraph{Subparagraph}
Useless.

\subsection{Figures}

The usage of vector graphics is highly recommendend. All figures are drawn using Microsoft Visio with the IAS template. In this way you can easily generate PDFs.
Figures are included with (Fig.~\ref{fig:testbild}) the following statement:
\begin{figure}[htbp!] % Floating Objekt Bild, 
                     % Anordnungsvorschrift here, top, bottom, page
  \includegraphics[width = \textwidth]
        {adc-full-flash-block}
        \caption{ADC full flash converter.\label{fig:testbild}}
\end{figure}

\subsection{Abbreviations}
Abbreviations are introtuced as follows. After the usage of makeindex, they will also show up in the List of Abbreviations.  Digital-Analog-Converter (DAC) \abk{DAC}{Digital-Analog-Converter}.


\subsection{Cites and References}

Cite with \cite{Razavi00}. The Bibliography is generated from the file \enquote{mybib.bib} which can be edited with JabRef. This is a reference to a Standard \cite{ieee}.

These are examples for references to a certain chapter \ref{kap:einfuehrung} or page \pageref{kap:einfuehrung}.

\subsection{Equations and Units}

This is an example for an equation which can also be referenced \eqref{eq:u-kenn}.
\begin{equation}
  I_{aus} = U_{ein} \; \sqrt{2 I_{DC} K } \; \sqrt{1-\frac{K-2c}{2 I_{DC}}}, \; \;  K = \frac{W}{L} \frac{\beta}{2}
 \label{eq:u-kenn}
\end{equation}

An inline equation is done with $2 \cdot \sqrt{\sin(x)}$.

For units the  SIunitx package is highly recommended, e.g. \SI{1.2}{\um} or \si{\milli\ohm} instead of disgusting workarounds as 1\,$\mu$m.

\subsection{Confidential Version}
If some parts of your work are confidential, use the confidential switch, e.g. \confidential{This is confidential test.}{This is the non-confidential alternative.}

\subsection{Tables}

You can find further examples of tables with design guidelines at \cite{Mori07}. Avoid vertical lines. Tab. \ref{tab:linear-vergleich} and shows some examples.

\begin{table}[hbtp]
  \centering      
  \begin{tabular}{rrccc}
  \toprule
            &                 &             & konstante   &  Abh�ngigkeit\\
            & Konzept         & Linearit�t & Stromsumme   & von $U_{CM}$ \\
  \midrule
            & Differenzstufe   &              &             & \\
            & mit adaptiver   & mittel       & ja           & gering\\
            & Stromregelung   & & &\\
  \midrule
            & Gekoppelte       & hoch         & nein         & hoch\\
            & Transistorpaare &             & &\\
  \midrule
            & Paar mit stab.   & mittel       & ja           & sehr gering \\
            & Gleichtaktlage   &             & &\\
  \midrule
  & Unges�ttigtes             & sehr hoch   & nein         & hoch \\
  & Transistorpaar             & & &\\
  \bottomrule
  \end{tabular}
        
   \caption{Comparsion of linear amplifiers\label{tab:linear-vergleich}}   
\end{table}


\subsection{Items}

Example of items:
\begin{itemize}
        \item Punkt 1
        \item Punkt 2
  \item[a)] Hier ein anderes Aufz�hlungszeicheen

\end{itemize}
The itemize environment also supports hierarchy.
�
��


\subsection{Listings}
The usage of sourcecode listings is shown for a VHDL(VHDL-AMS) listing \ref{lis:vhdl} and a MATLAB listing \ref{lis:matlab}. It is recommended to use an extra file for the sourcecode.

\begin{lstlisting}[language={[AMS]VHDL}, caption=VHDL code example.,label=lis:vhdl]
library ieee;
  use ieee.std_logic_1164.all;

entity dig_clock is
  port (clk : out std_logic);               -- clock output
end entity dig_clock;

architecture QuickStart of dig_clock is
begin
...
end architecture QuickStart;
\end{lstlisting}

\begin{lstlisting}[language=Matlab, caption=MATLAB example., label=lis:matlab]
clear;
fs=1;
f=(0:fs/10000:fs/2);
figure
plot([0 fs/2],[1 1],'k-');
hold on;
k = 1;
plot(f,2^k*(sin(2*pi*f/2/fs)).^k,'b:');
k = 2;
hold on;
plot(f,2^k*(sin(2*pi*f/2/fs)).^k,'g--');
k = 3;
hold on;
plot(f,2^k*(sin(2*pi*f/2/fs)).^k,'r-.');
grid on;
ylabel('|H_{k}(f)|','FontSize',14);
xlabel('f/f_{s}','FontSize',14)
set(gca,'FontSize',14);
\end{lstlisting}

\subsection{Plots}
To draw nice Plots of simulations you can use PGFplots and TIKZpicture.
Please refer to \href{http://pgfplots.sourceforge.net/pgfplots.pdf}{PGF Manual} and \href{http://www.texample.net/tikz/examples/}{TEXample.net}.
\begin{figure}[!h]
\centering
\begin{tikzpicture}
\begin{axis}[width=0.95\textwidth, height=5cm, 
			xlabel={Some Data},
			ylabel={Some Value},
			xmin=-1,
      xmax=7,
      ymin=-2,
      ymax=10
      ]
%\addplot[blue, ultra thin] (x,x*x);
%\addplot[red,  ultra thick] (x*x,x);
\addplot[black, mark=*, domain= 1.5:5 ] table [ x=a, y=c, col sep=comma] {data.csv};
\end{axis}
\end{tikzpicture}
\caption{My Plot} \label{fig:My Plot}
\end{figure}